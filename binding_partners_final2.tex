\documentclass[]{article}
\usepackage{lmodern}
\usepackage{amssymb,amsmath}
\usepackage{ifxetex,ifluatex}
\usepackage{fixltx2e} % provides \textsubscript
\ifnum 0\ifxetex 1\fi\ifluatex 1\fi=0 % if pdftex
  \usepackage[T1]{fontenc}
  \usepackage[utf8]{inputenc}
\else % if luatex or xelatex
  \ifxetex
    \usepackage{mathspec}
  \else
    \usepackage{fontspec}
  \fi
  \defaultfontfeatures{Ligatures=TeX,Scale=MatchLowercase}
\fi
% use upquote if available, for straight quotes in verbatim environments
\IfFileExists{upquote.sty}{\usepackage{upquote}}{}
% use microtype if available
\IfFileExists{microtype.sty}{%
\usepackage{microtype}
\UseMicrotypeSet[protrusion]{basicmath} % disable protrusion for tt fonts
}{}
\usepackage[margin=1in]{geometry}
\usepackage{hyperref}
\hypersetup{unicode=true,
            pdftitle={binding\_partners\_final2.R},
            pdfauthor={patrickjonker},
            pdfborder={0 0 0},
            breaklinks=true}
\urlstyle{same}  % don't use monospace font for urls
\usepackage{color}
\usepackage{fancyvrb}
\newcommand{\VerbBar}{|}
\newcommand{\VERB}{\Verb[commandchars=\\\{\}]}
\DefineVerbatimEnvironment{Highlighting}{Verbatim}{commandchars=\\\{\}}
% Add ',fontsize=\small' for more characters per line
\usepackage{framed}
\definecolor{shadecolor}{RGB}{248,248,248}
\newenvironment{Shaded}{\begin{snugshade}}{\end{snugshade}}
\newcommand{\AlertTok}[1]{\textcolor[rgb]{0.94,0.16,0.16}{#1}}
\newcommand{\AnnotationTok}[1]{\textcolor[rgb]{0.56,0.35,0.01}{\textbf{\textit{#1}}}}
\newcommand{\AttributeTok}[1]{\textcolor[rgb]{0.77,0.63,0.00}{#1}}
\newcommand{\BaseNTok}[1]{\textcolor[rgb]{0.00,0.00,0.81}{#1}}
\newcommand{\BuiltInTok}[1]{#1}
\newcommand{\CharTok}[1]{\textcolor[rgb]{0.31,0.60,0.02}{#1}}
\newcommand{\CommentTok}[1]{\textcolor[rgb]{0.56,0.35,0.01}{\textit{#1}}}
\newcommand{\CommentVarTok}[1]{\textcolor[rgb]{0.56,0.35,0.01}{\textbf{\textit{#1}}}}
\newcommand{\ConstantTok}[1]{\textcolor[rgb]{0.00,0.00,0.00}{#1}}
\newcommand{\ControlFlowTok}[1]{\textcolor[rgb]{0.13,0.29,0.53}{\textbf{#1}}}
\newcommand{\DataTypeTok}[1]{\textcolor[rgb]{0.13,0.29,0.53}{#1}}
\newcommand{\DecValTok}[1]{\textcolor[rgb]{0.00,0.00,0.81}{#1}}
\newcommand{\DocumentationTok}[1]{\textcolor[rgb]{0.56,0.35,0.01}{\textbf{\textit{#1}}}}
\newcommand{\ErrorTok}[1]{\textcolor[rgb]{0.64,0.00,0.00}{\textbf{#1}}}
\newcommand{\ExtensionTok}[1]{#1}
\newcommand{\FloatTok}[1]{\textcolor[rgb]{0.00,0.00,0.81}{#1}}
\newcommand{\FunctionTok}[1]{\textcolor[rgb]{0.00,0.00,0.00}{#1}}
\newcommand{\ImportTok}[1]{#1}
\newcommand{\InformationTok}[1]{\textcolor[rgb]{0.56,0.35,0.01}{\textbf{\textit{#1}}}}
\newcommand{\KeywordTok}[1]{\textcolor[rgb]{0.13,0.29,0.53}{\textbf{#1}}}
\newcommand{\NormalTok}[1]{#1}
\newcommand{\OperatorTok}[1]{\textcolor[rgb]{0.81,0.36,0.00}{\textbf{#1}}}
\newcommand{\OtherTok}[1]{\textcolor[rgb]{0.56,0.35,0.01}{#1}}
\newcommand{\PreprocessorTok}[1]{\textcolor[rgb]{0.56,0.35,0.01}{\textit{#1}}}
\newcommand{\RegionMarkerTok}[1]{#1}
\newcommand{\SpecialCharTok}[1]{\textcolor[rgb]{0.00,0.00,0.00}{#1}}
\newcommand{\SpecialStringTok}[1]{\textcolor[rgb]{0.31,0.60,0.02}{#1}}
\newcommand{\StringTok}[1]{\textcolor[rgb]{0.31,0.60,0.02}{#1}}
\newcommand{\VariableTok}[1]{\textcolor[rgb]{0.00,0.00,0.00}{#1}}
\newcommand{\VerbatimStringTok}[1]{\textcolor[rgb]{0.31,0.60,0.02}{#1}}
\newcommand{\WarningTok}[1]{\textcolor[rgb]{0.56,0.35,0.01}{\textbf{\textit{#1}}}}
\usepackage{graphicx,grffile}
\makeatletter
\def\maxwidth{\ifdim\Gin@nat@width>\linewidth\linewidth\else\Gin@nat@width\fi}
\def\maxheight{\ifdim\Gin@nat@height>\textheight\textheight\else\Gin@nat@height\fi}
\makeatother
% Scale images if necessary, so that they will not overflow the page
% margins by default, and it is still possible to overwrite the defaults
% using explicit options in \includegraphics[width, height, ...]{}
\setkeys{Gin}{width=\maxwidth,height=\maxheight,keepaspectratio}
\IfFileExists{parskip.sty}{%
\usepackage{parskip}
}{% else
\setlength{\parindent}{0pt}
\setlength{\parskip}{6pt plus 2pt minus 1pt}
}
\setlength{\emergencystretch}{3em}  % prevent overfull lines
\providecommand{\tightlist}{%
  \setlength{\itemsep}{0pt}\setlength{\parskip}{0pt}}
\setcounter{secnumdepth}{0}
% Redefines (sub)paragraphs to behave more like sections
\ifx\paragraph\undefined\else
\let\oldparagraph\paragraph
\renewcommand{\paragraph}[1]{\oldparagraph{#1}\mbox{}}
\fi
\ifx\subparagraph\undefined\else
\let\oldsubparagraph\subparagraph
\renewcommand{\subparagraph}[1]{\oldsubparagraph{#1}\mbox{}}
\fi

%%% Use protect on footnotes to avoid problems with footnotes in titles
\let\rmarkdownfootnote\footnote%
\def\footnote{\protect\rmarkdownfootnote}

%%% Change title format to be more compact
\usepackage{titling}

% Create subtitle command for use in maketitle
\providecommand{\subtitle}[1]{
  \posttitle{
    \begin{center}\large#1\end{center}
    }
}

\setlength{\droptitle}{-2em}

  \title{binding\_partners\_final2.R}
    \pretitle{\vspace{\droptitle}\centering\huge}
  \posttitle{\par}
    \author{patrickjonker}
    \preauthor{\centering\large\emph}
  \postauthor{\par}
      \predate{\centering\large\emph}
  \postdate{\par}
    \date{2020-03-20}


\begin{document}
\maketitle

\begin{Shaded}
\begin{Highlighting}[]
\CommentTok{##Description of the project}

\CommentTok{##I am rotating in a Cancer Biology lab that studies the protein Calreticulin (CALR). }
\CommentTok{##We see that mutant calreticulin drives a cancer phenotype in myeloid cells.}
\CommentTok{##We currently have proteomics data showing all of the proteins mutant calreticulin binds }
\CommentTok{##in the cell (which is a lot of proteins), but have done nothing with it. The code below, as well}
\CommentTok{##as code run prior (in python) attempts to take those data and overlap them with 15 datasets}
\CommentTok{##identifying genes in different pathways. This way, by the end of this exercise we will have}
\CommentTok{##novel information concerning which proteins mutant calreticulin is binding. }
\CommentTok{##The goals of this project include producing multiple graphs summarizing my findings, as well}
\CommentTok{##as a few summative tables indicating key genes (and their protein products) whose}
\CommentTok{##functions are potentially altered due to mutant calreticulin binding. A general note : most of}
\CommentTok{##the code below was run multiple times--once for each data set. I have tried to group these }
\CommentTok{##repeated bits of code clearly, so that you can read one line, get the picture, and move on.}


\KeywordTok{library}\NormalTok{(tidyr)}
\KeywordTok{library}\NormalTok{(tibble)}
\KeywordTok{library}\NormalTok{(ggplot2)}
\KeywordTok{library}\NormalTok{(Stack)}
\KeywordTok{library}\NormalTok{(dplyr)}
\end{Highlighting}
\end{Shaded}

\begin{verbatim}
## 
## Attaching package: 'dplyr'
\end{verbatim}

\begin{verbatim}
## The following objects are masked from 'package:stats':
## 
##     filter, lag
\end{verbatim}

\begin{verbatim}
## The following objects are masked from 'package:base':
## 
##     intersect, setdiff, setequal, union
\end{verbatim}

\begin{Shaded}
\begin{Highlighting}[]
\CommentTok{## first find upr gene overlap data (below)}
\KeywordTok{setwd}\NormalTok{(}\StringTok{"~/Documents/Graduate_School/Research/Elf_Research/Binding_Partners"}\NormalTok{)}
\NormalTok{binding_partners <-}\StringTok{ }\KeywordTok{read.delim}\NormalTok{(}\StringTok{"binding_partners.txt"}\NormalTok{)}
\KeywordTok{length}\NormalTok{(binding_partners)}
\end{Highlighting}
\end{Shaded}

\begin{verbatim}
## [1] 12
\end{verbatim}

\begin{Shaded}
\begin{Highlighting}[]
\KeywordTok{View}\NormalTok{(binding_partners)}
\NormalTok{upr_proteins <-}\StringTok{ }\KeywordTok{read.csv}\NormalTok{(}\StringTok{'upr_proteins.csv'}\NormalTok{, }\DataTypeTok{header =} \OtherTok{FALSE}\NormalTok{)}
\KeywordTok{view}\NormalTok{(upr_proteins)}
\NormalTok{v1 <-}\StringTok{ }\NormalTok{upr_proteins}\OperatorTok{$}\NormalTok{V1}
\NormalTok{v2 <-}\StringTok{ }\NormalTok{binding_partners}\OperatorTok{$}\NormalTok{gene}
\NormalTok{upr_overlay <-}\StringTok{ }\KeywordTok{intersect}\NormalTok{(v1, v2)}
\CommentTok{##upr gene overlap data end}



\CommentTok{##NFkB signaling protein overlap}
\NormalTok{NFkB_raw <-}\StringTok{ }\KeywordTok{read.csv}\NormalTok{(}\StringTok{"NFkB_proteins.csv"}\NormalTok{, }\DataTypeTok{header =} \OtherTok{FALSE}\NormalTok{)}

\NormalTok{NFkB_proteins <-}\StringTok{ }\KeywordTok{t}\NormalTok{(NFkB_raw)}
\KeywordTok{view}\NormalTok{(NFkB_proteins)}
\NormalTok{NFkB_proteins <-}\StringTok{ }\KeywordTok{as.data.frame}\NormalTok{(NFkB_proteins)}
\CommentTok{##change data from atomic list to recursive using as.data.frame function}
\NormalTok{v3 <-}\StringTok{ }\NormalTok{NFkB_proteins}\OperatorTok{$}\NormalTok{V1}
\NormalTok{NFkB_overlap <-}\StringTok{ }\KeywordTok{intersect}\NormalTok{(v2, v3)}
\KeywordTok{view}\NormalTok{(NFkB_overlap)}
\CommentTok{##NFkB overlap data ends}




\CommentTok{##Amino Acid transporter overlap}
\NormalTok{AATransporter_raw <-}\StringTok{ }\KeywordTok{read.csv}\NormalTok{(}\StringTok{"AATransporter_proteins.csv"}\NormalTok{, }\DataTypeTok{header =} \OtherTok{FALSE}\NormalTok{)}
\KeywordTok{view}\NormalTok{(AATransporter_raw)}
\NormalTok{AATransporter_proteins <-}\StringTok{ }\KeywordTok{t}\NormalTok{(AATransporter_raw)}
\KeywordTok{view}\NormalTok{(AATransporter_proteins)}
\NormalTok{AATransporter_proteins <-}\StringTok{ }\KeywordTok{as.data.frame}\NormalTok{(AATransporter_proteins)}
\CommentTok{##change data from atomic list to recursive using as.data.frame function}
\NormalTok{v4 <-}\StringTok{ }\NormalTok{AATransporter_proteins}\OperatorTok{$}\NormalTok{V1}
\NormalTok{AATransporter_overlap <-}\StringTok{ }\KeywordTok{intersect}\NormalTok{(v2, v4)}
\KeywordTok{view}\NormalTok{(AATransporter_overlap)}
\CommentTok{##Amino Acid transporter overlap end}



\CommentTok{##Antigen processing}
\NormalTok{Antigen_processing_raw <-}\StringTok{ }\KeywordTok{read.csv}\NormalTok{(}\StringTok{"Antigen_processing_proteins.csv"}\NormalTok{, }\DataTypeTok{header =} \OtherTok{FALSE}\NormalTok{)}
\KeywordTok{view}\NormalTok{(Antigen_processing_raw)}
\NormalTok{Antigen_processing_proteins <-}\StringTok{ }\KeywordTok{t}\NormalTok{(Antigen_processing_raw)}
\KeywordTok{view}\NormalTok{(Antigen_processing_proteins)}
\NormalTok{Antigen_processing_proteins <-}\StringTok{ }\KeywordTok{as.data.frame}\NormalTok{(Antigen_processing_proteins)}
\CommentTok{##change data from atomic list to recursive using as.data.frame function}
\NormalTok{v5 <-}\StringTok{ }\NormalTok{Antigen_processing_proteins}\OperatorTok{$}\NormalTok{V1}
\NormalTok{Antigen_processing_overlap <-}\StringTok{ }\KeywordTok{intersect}\NormalTok{(v2, v5)}
\KeywordTok{view}\NormalTok{(Antigen_processing_overlap)}
\CommentTok{##antigen processing end}




\CommentTok{##Apoptosis proteins start}
\NormalTok{Apoptosis_raw <-}\StringTok{ }\KeywordTok{read.csv}\NormalTok{(}\StringTok{"Apoptosis_proteins.csv"}\NormalTok{, }\DataTypeTok{header =} \OtherTok{FALSE}\NormalTok{)}
\KeywordTok{view}\NormalTok{(Apoptosis_raw)}
\NormalTok{Apoptosis_proteins <-}\StringTok{ }\KeywordTok{t}\NormalTok{(Apoptosis_raw)}
\KeywordTok{view}\NormalTok{(Apoptosis_proteins)}
\NormalTok{Apoptosis_proteins <-}\StringTok{ }\KeywordTok{as.data.frame}\NormalTok{(Apoptosis_proteins)}
\CommentTok{##change data from atomic list to recursive using as.data.frame function}
\NormalTok{v6 <-}\StringTok{ }\NormalTok{Apoptosis_proteins}\OperatorTok{$}\NormalTok{V1}
\NormalTok{Apoptosis_overlap <-}\StringTok{ }\KeywordTok{intersect}\NormalTok{(v2, v6)}
\KeywordTok{view}\NormalTok{(Apoptosis_overlap)}
\CommentTok{##Apoptosis end}




\CommentTok{##}
\NormalTok{Protein_folding_raw <-}\StringTok{ }\KeywordTok{read.csv}\NormalTok{(}\StringTok{"Protein_folding.csv"}\NormalTok{, }\DataTypeTok{header =} \OtherTok{FALSE}\NormalTok{)}
\KeywordTok{view}\NormalTok{(Protein_folding_raw)}
\NormalTok{Protein_folding <-}\StringTok{ }\KeywordTok{t}\NormalTok{(Protein_folding_raw)}
\KeywordTok{view}\NormalTok{(Protein_folding)}
\NormalTok{Protein_folding <-}\StringTok{ }\KeywordTok{as.data.frame}\NormalTok{(Protein_folding)}
\CommentTok{##change data from atomic list to recursive using as.data.frame function}
\NormalTok{v7 <-}\StringTok{ }\NormalTok{Protein_folding}\OperatorTok{$}\NormalTok{V1}
\NormalTok{Protein_folding_overlap <-}\StringTok{ }\KeywordTok{intersect}\NormalTok{(v2, v7)}
\KeywordTok{view}\NormalTok{(Protein_folding_overlap)}
\CommentTok{##}





\CommentTok{##PD1 proteins}
\NormalTok{PD1_raw <-}\StringTok{ }\KeywordTok{read.csv}\NormalTok{(}\StringTok{"PD1_proteins.csv"}\NormalTok{, }\DataTypeTok{header =} \OtherTok{FALSE}\NormalTok{)}
\KeywordTok{view}\NormalTok{(PD1_raw)}
\NormalTok{PD1_proteins <-}\StringTok{ }\KeywordTok{t}\NormalTok{(PD1_raw)}
\KeywordTok{view}\NormalTok{(PD1_proteins)}
\NormalTok{PD1_proteins <-}\StringTok{ }\KeywordTok{as.data.frame}\NormalTok{(PD1_proteins)}
\CommentTok{##change data from atomic list to recursive using as.data.frame function}
\NormalTok{v8 <-}\StringTok{ }\NormalTok{PD1_proteins}\OperatorTok{$}\NormalTok{V1}
\NormalTok{PD1_protein_overlap <-}\StringTok{ }\KeywordTok{intersect}\NormalTok{(v2, v8)}
\KeywordTok{view}\NormalTok{(PD1_protein_overlap)}
\CommentTok{##end}





\CommentTok{##p53 independent DNA damage/repair proteins}
\NormalTok{p53_raw <-}\StringTok{ }\KeywordTok{read.csv}\NormalTok{(}\StringTok{"p53_indep_DNA_damage.csv"}\NormalTok{, }\DataTypeTok{header =} \OtherTok{FALSE}\NormalTok{)}
\KeywordTok{view}\NormalTok{(p53_raw)}
\NormalTok{p53_indep_DNA_damage <-}\StringTok{ }\KeywordTok{t}\NormalTok{(p53_raw)}
\KeywordTok{view}\NormalTok{(p53_indep_DNA_damage)}
\NormalTok{p53_indep_DNA_damage<-}\StringTok{ }\KeywordTok{as.data.frame}\NormalTok{(p53_indep_DNA_damage)}
\CommentTok{##change data from atomic list to recursive using as.data.frame function}
\NormalTok{v9 <-}\StringTok{ }\NormalTok{p53_indep_DNA_damage}\OperatorTok{$}\NormalTok{V1}
\NormalTok{p53_indep_overlap <-}\StringTok{ }\KeywordTok{intersect}\NormalTok{(v2, v9)}
\KeywordTok{view}\NormalTok{(p53_indep_overlap)}
\CommentTok{##end}





\CommentTok{##nucleosome proteins}
\NormalTok{nucleosome_raw <-}\StringTok{ }\KeywordTok{read.csv}\NormalTok{(}\StringTok{"nucleosome_proteins.csv"}\NormalTok{, }\DataTypeTok{header =} \OtherTok{FALSE}\NormalTok{)}
\KeywordTok{view}\NormalTok{(nucleosome_raw)}
\NormalTok{nucleosome_proteins <-}\StringTok{ }\KeywordTok{t}\NormalTok{(nucleosome_raw)}
\KeywordTok{view}\NormalTok{(nucleosome_proteins)}
\NormalTok{nucleosome_proteins <-}\StringTok{ }\KeywordTok{as.data.frame}\NormalTok{(nucleosome_proteins)}
\CommentTok{##change data from atomic list to recursive using as.data.frame function}
\NormalTok{v10 <-}\StringTok{ }\NormalTok{nucleosome_proteins}\OperatorTok{$}\NormalTok{V1}
\NormalTok{nucleosome_proteins_overlap <-}\StringTok{ }\KeywordTok{intersect}\NormalTok{(v2, v10)}
\KeywordTok{view}\NormalTok{(nucleosome_proteins_overlap)}
\CommentTok{##nucleosome proteins end}





\CommentTok{##}
\NormalTok{mTOR_raw <-}\StringTok{ }\KeywordTok{read.csv}\NormalTok{(}\StringTok{"mTOR_proteins.csv"}\NormalTok{, }\DataTypeTok{header =} \OtherTok{FALSE}\NormalTok{)}
\KeywordTok{view}\NormalTok{(mTOR_raw)}
\NormalTok{mTOR_proteins <-}\StringTok{ }\KeywordTok{t}\NormalTok{(mTOR_raw)}
\KeywordTok{view}\NormalTok{(mTOR_proteins)}
\NormalTok{mTOR_proteins <-}\StringTok{ }\KeywordTok{as.data.frame}\NormalTok{(mTOR_proteins)}
\CommentTok{##change data from atomic list to recursive using as.data.frame function}
\NormalTok{v11 <-}\StringTok{ }\NormalTok{mTOR_proteins}\OperatorTok{$}\NormalTok{V1}
\NormalTok{mTOR_proteins_overlap <-}\StringTok{ }\KeywordTok{intersect}\NormalTok{(v2, v11)}
\KeywordTok{view}\NormalTok{(mTOR_proteins_overlap)}
\CommentTok{##}






\CommentTok{##Mitotic telophase/cytokenesis proteins}
\NormalTok{Mitotic_raw <-}\StringTok{ }\KeywordTok{read.csv}\NormalTok{(}\StringTok{"Mitotic_proteins.csv"}\NormalTok{, }\DataTypeTok{header =} \OtherTok{FALSE}\NormalTok{)}
\KeywordTok{view}\NormalTok{(Mitotic_raw)}
\NormalTok{Mitotic_proteins <-}\StringTok{ }\KeywordTok{t}\NormalTok{(Mitotic_raw)}
\KeywordTok{view}\NormalTok{(Mitotic_proteins)}
\NormalTok{Mitotic_proteins <-}\StringTok{ }\KeywordTok{as.data.frame}\NormalTok{(Mitotic_proteins)}
\CommentTok{##change data from atomic list to recursive using as.data.frame function}
\NormalTok{v12 <-}\StringTok{ }\NormalTok{Mitotic_proteins}\OperatorTok{$}\NormalTok{V1}
\NormalTok{Mitotic_proteins_overlap <-}\StringTok{ }\KeywordTok{intersect}\NormalTok{(v2, v12)}
\KeywordTok{view}\NormalTok{(Mitotic_proteins_overlap)}
\CommentTok{##end}






\CommentTok{##Amino Acid Metabolism}
\NormalTok{AAMetabolism_raw <-}\StringTok{ }\KeywordTok{read.csv}\NormalTok{(}\StringTok{"AAMetabolism.csv"}\NormalTok{, }\DataTypeTok{header =} \OtherTok{FALSE}\NormalTok{)}
\KeywordTok{view}\NormalTok{(AAMetabolism_raw)}
\NormalTok{AAMetabolism_proteins <-}\StringTok{ }\KeywordTok{t}\NormalTok{(AAMetabolism_raw)}
\KeywordTok{view}\NormalTok{(AAMetabolism_proteins)}
\NormalTok{AAMetabolism_proteins <-}\StringTok{ }\KeywordTok{as.data.frame}\NormalTok{(AAMetabolism_proteins)}
\CommentTok{##change data from atomic list to recursive using as.data.frame function}
\NormalTok{v13 <-}\StringTok{ }\NormalTok{AAMetabolism_proteins}\OperatorTok{$}\NormalTok{V1}
\NormalTok{AAMetabolism_proteins_overlap <-}\StringTok{ }\KeywordTok{intersect}\NormalTok{(v2, v13)}
\KeywordTok{view}\NormalTok{(AAMetabolism_proteins_overlap)}
\CommentTok{##End}




\CommentTok{##Glycolysis proteins}
\NormalTok{Glycolysis_raw <-}\StringTok{ }\KeywordTok{read.csv}\NormalTok{(}\StringTok{"Glycolysis_proteins.csv"}\NormalTok{, }\DataTypeTok{header =} \OtherTok{FALSE}\NormalTok{)}
\KeywordTok{view}\NormalTok{(Glycolysis_raw)}
\NormalTok{Glycolysis_proteins <-}\StringTok{ }\KeywordTok{t}\NormalTok{(Glycolysis_raw)}
\KeywordTok{view}\NormalTok{(Glycolysis_proteins)}
\NormalTok{Glycolysis_proteins <-}\StringTok{ }\KeywordTok{as.data.frame}\NormalTok{(Glycolysis_proteins)}
\CommentTok{##change data from atomic list to recursive using as.data.frame function}
\NormalTok{v14 <-}\StringTok{ }\NormalTok{Glycolysis_proteins}\OperatorTok{$}\NormalTok{V1}
\NormalTok{Glycolysis_proteins_overlap <-}\StringTok{ }\KeywordTok{intersect}\NormalTok{(v2, v14)}
\KeywordTok{view}\NormalTok{(Glycolysis_proteins_overlap)}
\CommentTok{##End}




\CommentTok{##TCR signaling proteins}
\NormalTok{TCR_raw <-}\StringTok{ }\KeywordTok{read.csv}\NormalTok{(}\StringTok{"TCR_signaling.csv"}\NormalTok{, }\DataTypeTok{header =} \OtherTok{FALSE}\NormalTok{)}
\KeywordTok{view}\NormalTok{(TCR_raw)}
\NormalTok{TCR_signaling_proteins <-}\StringTok{ }\KeywordTok{t}\NormalTok{(TCR_raw)}
\KeywordTok{view}\NormalTok{(TCR_signaling_proteins)}
\NormalTok{TCR_signaling_proteins <-}\StringTok{ }\KeywordTok{as.data.frame}\NormalTok{(TCR_signaling_proteins)}
\CommentTok{##change data from atomic list to recursive using as.data.frame function}
\NormalTok{v15 <-}\StringTok{ }\NormalTok{TCR_signaling_proteins}\OperatorTok{$}\NormalTok{V1}
\NormalTok{TCR_proteins_overlap <-}\StringTok{ }\KeywordTok{intersect}\NormalTok{(v2, v15)}
\KeywordTok{view}\NormalTok{(TCR_proteins_overlap)}
\CommentTok{##}

\CommentTok{##ATF4 binding proteins}
\NormalTok{ATF4_raw <-}\StringTok{ }\KeywordTok{read.csv}\NormalTok{(}\StringTok{"ATF4_binding.csv"}\NormalTok{, }\DataTypeTok{header =} \OtherTok{FALSE}\NormalTok{)}
\KeywordTok{view}\NormalTok{(ATF4_raw)}
\NormalTok{ATF4_binding_proteins <-}\StringTok{ }\KeywordTok{t}\NormalTok{(ATF4_raw)}
\KeywordTok{view}\NormalTok{(ATF4_binding_proteins)}
\NormalTok{ATF4_binding_proteins <-}\StringTok{ }\KeywordTok{as.data.frame}\NormalTok{(ATF4_binding_proteins)}
\CommentTok{##change data from atomic list to recursive using as.data.frame function}
\NormalTok{v16 <-}\StringTok{ }\NormalTok{ATF4_binding_proteins}\OperatorTok{$}\NormalTok{V1}
\NormalTok{ATF4_binding_overlap <-}\StringTok{ }\KeywordTok{intersect}\NormalTok{(v2, v16)}
\KeywordTok{view}\NormalTok{(ATF4_binding_overlap)}
\CommentTok{##}

\CommentTok{##Match common genes with values to get frequencies by finding their place in the}
\CommentTok{##data frame using the "match" function}


\CommentTok{#TCR}
\NormalTok{TCR_numbers <-}\StringTok{ }\KeywordTok{match}\NormalTok{(TCR_proteins_overlap, binding_partners}\OperatorTok{$}\NormalTok{gene, }\DataTypeTok{nomatch =} \OtherTok{FALSE}\NormalTok{)}
\NormalTok{TCR_frequency <-}\StringTok{ }\KeywordTok{mean}\NormalTok{((binding_partners}\OperatorTok{$}\NormalTok{frequency[TCR_numbers]))}

\CommentTok{#Amino acid transport}
\NormalTok{AATransport_numbers <-}\StringTok{ }\KeywordTok{match}\NormalTok{(AATransporter_overlap, binding_partners}\OperatorTok{$}\NormalTok{gene, }\DataTypeTok{nomatch =} \OtherTok{FALSE}\NormalTok{)}
\NormalTok{AATransport_frequency <-}\StringTok{ }\KeywordTok{mean}\NormalTok{(binding_partners}\OperatorTok{$}\NormalTok{frequency[AATransport_numbers])}

\CommentTok{#Amino acid metabolism}
\NormalTok{AAMetabolism_numbers <-}\StringTok{ }\KeywordTok{match}\NormalTok{(AAMetabolism_proteins_overlap, binding_partners}\OperatorTok{$}\NormalTok{gene, }\DataTypeTok{nomatch =} \OtherTok{FALSE}\NormalTok{)}
\NormalTok{AAMetabolism_frequency <-}\StringTok{ }\KeywordTok{mean}\NormalTok{(binding_partners}\OperatorTok{$}\NormalTok{frequency[AAMetabolism_numbers])}

\CommentTok{#Antigen processing}
\NormalTok{Antigen_processing_numbers <-}\StringTok{ }\KeywordTok{match}\NormalTok{(Antigen_processing_overlap, binding_partners}\OperatorTok{$}\NormalTok{gene, }\DataTypeTok{nomatch =} \OtherTok{FALSE}\NormalTok{)}
\NormalTok{Antigen_processing_frequency <-}\StringTok{ }\KeywordTok{mean}\NormalTok{(binding_partners}\OperatorTok{$}\NormalTok{frequency[Antigen_processing_numbers])}

\CommentTok{#Apoptosis}
\NormalTok{Apoptosis_numbers <-}\StringTok{ }\KeywordTok{match}\NormalTok{(Apoptosis_overlap, binding_partners}\OperatorTok{$}\NormalTok{gene, }\DataTypeTok{nomatch =} \OtherTok{FALSE}\NormalTok{)}
\NormalTok{Apoptosis_frequency <-}\StringTok{ }\KeywordTok{mean}\NormalTok{(binding_partners}\OperatorTok{$}\NormalTok{frequency[Apoptosis_numbers])}

\CommentTok{#ATF4}
\NormalTok{ATF4_numbers <-}\StringTok{ }\KeywordTok{match}\NormalTok{(ATF4_binding_overlap, binding_partners}\OperatorTok{$}\NormalTok{gene, }\DataTypeTok{nomatch =} \OtherTok{FALSE}\NormalTok{)}
\NormalTok{ATF4_frequency <-}\StringTok{ }\KeywordTok{mean}\NormalTok{(binding_partners}\OperatorTok{$}\NormalTok{frequency[ATF4_numbers])}

\CommentTok{#Glycolysis}
\NormalTok{Glycolysis_numbers <-}\StringTok{ }\KeywordTok{match}\NormalTok{(Glycolysis_proteins_overlap, binding_partners}\OperatorTok{$}\NormalTok{gene, }\DataTypeTok{nomatch =} \OtherTok{FALSE}\NormalTok{)}
\NormalTok{Glycolysis_frequency <-}\StringTok{ }\KeywordTok{mean}\NormalTok{(binding_partners}\OperatorTok{$}\NormalTok{frequency[Glycolysis_numbers])}

\CommentTok{#Mitotic proteins}
\NormalTok{Mitotic_numbers <-}\StringTok{ }\KeywordTok{match}\NormalTok{(Mitotic_proteins_overlap, binding_partners}\OperatorTok{$}\NormalTok{gene, }\DataTypeTok{nomatch =} \OtherTok{FALSE}\NormalTok{)}
\NormalTok{Mitotic_frequency <-}\StringTok{ }\KeywordTok{mean}\NormalTok{(binding_partners}\OperatorTok{$}\NormalTok{frequency[Mitotic_numbers])}

\CommentTok{#mTOR }
\NormalTok{mTOR_numbers <-}\StringTok{ }\KeywordTok{match}\NormalTok{(mTOR_proteins_overlap, binding_partners}\OperatorTok{$}\NormalTok{gene, }\DataTypeTok{nomatch =} \OtherTok{FALSE}\NormalTok{)}
\NormalTok{mTOR_frequency <-}\StringTok{ }\KeywordTok{mean}\NormalTok{(binding_partners}\OperatorTok{$}\NormalTok{frequency[mTOR_numbers])}

\CommentTok{#NFkB}
\NormalTok{NFkB_numbers <-}\StringTok{ }\KeywordTok{match}\NormalTok{(NFkB_overlap, binding_partners}\OperatorTok{$}\NormalTok{gene, }\DataTypeTok{nomatch =} \OtherTok{FALSE}\NormalTok{)}
\NormalTok{NFkB_frequency <-}\StringTok{ }\KeywordTok{mean}\NormalTok{(binding_partners}\OperatorTok{$}\NormalTok{frequency[NFkB_numbers])}

\CommentTok{#nucleosome formation proteins}
\NormalTok{nucleosome_numbers <-}\StringTok{ }\KeywordTok{match}\NormalTok{(nucleosome_proteins_overlap, binding_partners}\OperatorTok{$}\NormalTok{gene, }\DataTypeTok{nomatch =} \OtherTok{FALSE}\NormalTok{)}
\NormalTok{nucleosome_frequency <-}\StringTok{ }\KeywordTok{mean}\NormalTok{(binding_partners}\OperatorTok{$}\NormalTok{frequency[nucleosome_numbers])}

\CommentTok{#p53 independent DNA Damage and repair}
\NormalTok{p53_numbers <-}\StringTok{ }\KeywordTok{match}\NormalTok{(p53_indep_overlap, binding_partners}\OperatorTok{$}\NormalTok{gene, }\DataTypeTok{nomatch =} \OtherTok{FALSE}\NormalTok{)}
\NormalTok{p53_frequency <-}\StringTok{ }\KeywordTok{mean}\NormalTok{(binding_partners}\OperatorTok{$}\NormalTok{frequency[p53_numbers])}

\CommentTok{#PD1 signaling proteins}
\NormalTok{PD1_numbers <-}\StringTok{ }\KeywordTok{match}\NormalTok{(PD1_protein_overlap, binding_partners}\OperatorTok{$}\NormalTok{gene, }\DataTypeTok{nomatch =} \OtherTok{FALSE}\NormalTok{)}
\NormalTok{PD1_frequency <-}\StringTok{ }\KeywordTok{mean}\NormalTok{(binding_partners}\OperatorTok{$}\NormalTok{frequency[PD1_numbers])}

\CommentTok{#protein folding proteins}
\NormalTok{Protein_folding_numbers <-}\StringTok{ }\KeywordTok{match}\NormalTok{(Protein_folding_overlap, binding_partners}\OperatorTok{$}\NormalTok{gene, }\DataTypeTok{nomatch =} \OtherTok{FALSE}\NormalTok{)}
\NormalTok{Protein_folding_frequency <-}\StringTok{ }\KeywordTok{mean}\NormalTok{(binding_partners}\OperatorTok{$}\NormalTok{frequency[Protein_folding_numbers])}

\CommentTok{#UPR proteins (unfolded protein response)}
\NormalTok{UPR_numbers <-}\StringTok{ }\KeywordTok{match}\NormalTok{(upr_overlay, binding_partners}\OperatorTok{$}\NormalTok{gene, }\DataTypeTok{nomatch =} \OtherTok{FALSE}\NormalTok{)}
\NormalTok{UPR_frequency <-}\StringTok{ }\KeywordTok{mean}\NormalTok{(binding_partners}\OperatorTok{$}\NormalTok{frequency[UPR_numbers])}
\CommentTok{#done calculating average frequency of each pathway analyzed}




\CommentTok{##Calculate the percent of each pathway bound by mutant CALR by dividing the number of}
\CommentTok{## proteins bound by the total number of proteins in the pathway}

\NormalTok{AAMetabolism_ratio <-}\StringTok{ }\KeywordTok{length}\NormalTok{(AAMetabolism_proteins_overlap)}\OperatorTok{/}\KeywordTok{length}\NormalTok{(AAMetabolism_proteins}\OperatorTok{$}\NormalTok{V1)}
\NormalTok{AATransport_ratio <-}\StringTok{ }\KeywordTok{length}\NormalTok{(AATransporter_overlap)}\OperatorTok{/}\KeywordTok{length}\NormalTok{(AATransporter_proteins}\OperatorTok{$}\NormalTok{V1)}
\NormalTok{Antigen_ratio <-}\StringTok{ }\KeywordTok{length}\NormalTok{(Antigen_processing_overlap)}\OperatorTok{/}\KeywordTok{length}\NormalTok{(Antigen_processing_proteins}\OperatorTok{$}\NormalTok{V1)}
\NormalTok{Apoptosis_ratio <-}\StringTok{ }\KeywordTok{length}\NormalTok{(Apoptosis_overlap)}\OperatorTok{/}\KeywordTok{length}\NormalTok{(Apoptosis_proteins}\OperatorTok{$}\NormalTok{V1)}
\NormalTok{ATF4_ratio <-}\StringTok{ }\KeywordTok{length}\NormalTok{(ATF4_binding_overlap)}\OperatorTok{/}\KeywordTok{length}\NormalTok{(ATF4_binding_proteins}\OperatorTok{$}\NormalTok{V1)}
\NormalTok{Mitotic_ratio <-}\StringTok{ }\KeywordTok{length}\NormalTok{(Mitotic_proteins_overlap)}\OperatorTok{/}\KeywordTok{length}\NormalTok{(Mitotic_proteins}\OperatorTok{$}\NormalTok{V1)}
\NormalTok{mTOR_ratio <-}\StringTok{ }\KeywordTok{length}\NormalTok{(mTOR_proteins_overlap)}\OperatorTok{/}\KeywordTok{length}\NormalTok{(mTOR_proteins}\OperatorTok{$}\NormalTok{V1)}
\NormalTok{NFkB_ratio <-}\StringTok{ }\KeywordTok{length}\NormalTok{(NFkB_overlap)}\OperatorTok{/}\KeywordTok{length}\NormalTok{(NFkB_proteins}\OperatorTok{$}\NormalTok{NFkB_proteins)}
\NormalTok{nucleosome_ratio <-}\StringTok{ }\KeywordTok{length}\NormalTok{(nucleosome_proteins_overlap)}\OperatorTok{/}\KeywordTok{length}\NormalTok{(nucleosome_proteins}\OperatorTok{$}\NormalTok{V1)}
\NormalTok{p53_ratio <-}\StringTok{ }\KeywordTok{length}\NormalTok{(p53_indep_overlap)}\OperatorTok{/}\KeywordTok{length}\NormalTok{(p53_indep_DNA_damage}\OperatorTok{$}\NormalTok{V1)}
\NormalTok{PD1_ratio <-}\StringTok{ }\KeywordTok{length}\NormalTok{(PD1_protein_overlap)}\OperatorTok{/}\KeywordTok{length}\NormalTok{(PD1_proteins}\OperatorTok{$}\NormalTok{V1)}
\NormalTok{Protein_folding_ratio <-}\StringTok{ }\KeywordTok{length}\NormalTok{(Protein_folding_overlap)}\OperatorTok{/}\KeywordTok{length}\NormalTok{(Protein_folding}\OperatorTok{$}\NormalTok{V1)}
\NormalTok{TCR_ratio <-}\StringTok{ }\KeywordTok{length}\NormalTok{(TCR_proteins_overlap)}\OperatorTok{/}\KeywordTok{length}\NormalTok{(TCR_signaling_proteins}\OperatorTok{$}\NormalTok{V1)}
\NormalTok{UPR_ratio <-}\StringTok{ }\KeywordTok{length}\NormalTok{(upr_overlay)}\OperatorTok{/}\KeywordTok{length}\NormalTok{(upr_proteins}\OperatorTok{$}\NormalTok{V1)}
\CommentTok{##}


\CommentTok{##compile each data set into overlap database that contains number of proteins##}
\CommentTok{##bound as well as the average frequencies (fold binding enrichment compared to wild type)}
\CommentTok{##of those select genes##}

\NormalTok{overlap <-}\StringTok{ }\KeywordTok{data.frame}\NormalTok{(}\StringTok{"Pathway"}\NormalTok{, }\StringTok{"Number of Genes"}\NormalTok{, }\StringTok{"Frequency"}\NormalTok{, }\StringTok{"Ratio"}\NormalTok{,}\DataTypeTok{stringsAsFactors =} \OtherTok{FALSE}\NormalTok{)}
\NormalTok{overlap <-}\StringTok{ }\KeywordTok{add_row}\NormalTok{(overlap, }\StringTok{"X.Pathway."}\NormalTok{ =}\StringTok{ "Amino Acid Metabolism"}\NormalTok{,}
                   \StringTok{"X.Number.of.Genes."}\NormalTok{ =}\StringTok{ }\KeywordTok{length}\NormalTok{(AAMetabolism_proteins_overlap),}
                   \StringTok{"X.Frequency."}\NormalTok{ =}\StringTok{ }\NormalTok{AAMetabolism_frequency, }\StringTok{"X.Ratio."}\NormalTok{ =}\StringTok{ }\NormalTok{AAMetabolism_ratio)}
\NormalTok{overlap <-}\StringTok{ }\KeywordTok{add_row}\NormalTok{(overlap, }\StringTok{"X.Pathway."}\NormalTok{ =}\StringTok{ "Amino Acid Transporters"}\NormalTok{,}
                   \StringTok{"X.Number.of.Genes."}\NormalTok{ =}\StringTok{ }\KeywordTok{length}\NormalTok{(AATransporter_overlap), }
                   \StringTok{"X.Frequency."}\NormalTok{ =}\StringTok{ }\NormalTok{AATransport_frequency,}\StringTok{"X.Ratio."}\NormalTok{ =}\StringTok{ }\NormalTok{AATransport_ratio)}
\NormalTok{overlap <-}\StringTok{ }\KeywordTok{add_row}\NormalTok{(overlap, }\StringTok{"X.Pathway."}\NormalTok{ =}\StringTok{ "Antigen processing"}\NormalTok{,}
                   \StringTok{"X.Number.of.Genes."}\NormalTok{ =}\StringTok{ }\KeywordTok{length}\NormalTok{(Antigen_processing_overlap), }
                   \StringTok{"X.Frequency."}\NormalTok{ =}\StringTok{ }\NormalTok{Antigen_processing_frequency, }\StringTok{"X.Ratio."}\NormalTok{ =}\StringTok{ }\NormalTok{Antigen_ratio)}
\NormalTok{overlap <-}\StringTok{ }\KeywordTok{add_row}\NormalTok{(overlap, }\StringTok{"X.Pathway."}\NormalTok{ =}\StringTok{ "Apoptosis"}\NormalTok{,}
                   \StringTok{"X.Number.of.Genes."}\NormalTok{ =}\StringTok{ }\KeywordTok{length}\NormalTok{(Apoptosis_overlap), }
                   \StringTok{"X.Frequency."}\NormalTok{ =}\StringTok{ }\NormalTok{Apoptosis_frequency, }\StringTok{"X.Ratio."}\NormalTok{ =}\StringTok{ }\NormalTok{Apoptosis_ratio)}
\NormalTok{overlap <-}\StringTok{ }\KeywordTok{add_row}\NormalTok{(overlap, }\StringTok{"X.Pathway."}\NormalTok{ =}\StringTok{ "ATF4 binding"}\NormalTok{,}
                   \StringTok{"X.Number.of.Genes."}\NormalTok{ =}\StringTok{ }\KeywordTok{length}\NormalTok{(ATF4_binding_overlap), }
                   \StringTok{"X.Frequency."}\NormalTok{ =}\StringTok{ }\NormalTok{ATF4_frequency, }\StringTok{"X.Ratio."}\NormalTok{ =}\StringTok{ }\NormalTok{ATF4_ratio)}
\NormalTok{overlap <-}\StringTok{ }\KeywordTok{add_row}\NormalTok{(overlap, }\StringTok{"X.Pathway."}\NormalTok{ =}\StringTok{ "Mitotic telophase/cytokenesis"}\NormalTok{,}
                   \StringTok{"X.Number.of.Genes."}\NormalTok{ =}\StringTok{ }\KeywordTok{length}\NormalTok{(Mitotic_proteins_overlap),}
                   \StringTok{"X.Frequency."}\NormalTok{ =}\StringTok{ }\NormalTok{Mitotic_frequency, }\StringTok{"X.Ratio."}\NormalTok{ =}\StringTok{ }\NormalTok{Mitotic_ratio)}
\NormalTok{overlap <-}\StringTok{ }\KeywordTok{add_row}\NormalTok{(overlap, }\StringTok{"X.Pathway."}\NormalTok{ =}\StringTok{ "mTOR"}\NormalTok{,}
                   \StringTok{"X.Number.of.Genes."}\NormalTok{ =}\StringTok{ }\KeywordTok{length}\NormalTok{(mTOR_proteins_overlap), }
                   \StringTok{"X.Frequency."}\NormalTok{ =}\StringTok{ }\NormalTok{mTOR_frequency, }\StringTok{"X.Ratio."}\NormalTok{ =}\StringTok{ }\NormalTok{mTOR_ratio)}
\NormalTok{overlap <-}\StringTok{ }\KeywordTok{add_row}\NormalTok{(overlap, }\StringTok{"X.Pathway."}\NormalTok{ =}\StringTok{ "NFkB"}\NormalTok{,}
                   \StringTok{"X.Number.of.Genes."}\NormalTok{ =}\StringTok{ }\KeywordTok{length}\NormalTok{(NFkB_overlap), }
                   \StringTok{"X.Frequency."}\NormalTok{ =}\StringTok{ }\NormalTok{NFkB_frequency, }\StringTok{"X.Ratio."}\NormalTok{ =}\StringTok{ }\NormalTok{NFkB_ratio)}
\NormalTok{overlap <-}\StringTok{ }\KeywordTok{add_row}\NormalTok{(overlap, }\StringTok{"X.Pathway."}\NormalTok{ =}\StringTok{ "Nucleosome"}\NormalTok{,}
                   \StringTok{"X.Number.of.Genes."}\NormalTok{ =}\StringTok{ }\KeywordTok{length}\NormalTok{(nucleosome_proteins_overlap), }
                   \StringTok{"X.Frequency."}\NormalTok{ =}\StringTok{ }\NormalTok{nucleosome_frequency, }\StringTok{"X.Ratio."}\NormalTok{ =}\StringTok{ }\NormalTok{nucleosome_ratio)}
\NormalTok{overlap <-}\StringTok{ }\KeywordTok{add_row}\NormalTok{(overlap, }\StringTok{"X.Pathway."}\NormalTok{ =}\StringTok{ "p53 independent DNA damage"}\NormalTok{,}
                   \StringTok{"X.Number.of.Genes."}\NormalTok{ =}\StringTok{ }\KeywordTok{length}\NormalTok{(p53_indep_overlap), }
                   \StringTok{"X.Frequency."}\NormalTok{ =}\StringTok{ }\NormalTok{p53_frequency, }\StringTok{"X.Ratio."}\NormalTok{ =}\StringTok{ }\NormalTok{p53_ratio)}
\NormalTok{overlap <-}\StringTok{ }\KeywordTok{add_row}\NormalTok{(overlap, }\StringTok{"X.Pathway."}\NormalTok{ =}\StringTok{ "PD1 signaling"}\NormalTok{,}
                   \StringTok{"X.Number.of.Genes."}\NormalTok{ =}\StringTok{ }\KeywordTok{length}\NormalTok{(PD1_protein_overlap),}
                   \StringTok{"X.Frequency."}\NormalTok{ =}\StringTok{ }\NormalTok{PD1_frequency, }\StringTok{"X.Ratio."}\NormalTok{ =}\StringTok{ }\NormalTok{PD1_ratio)}
\NormalTok{overlap <-}\StringTok{ }\KeywordTok{add_row}\NormalTok{(overlap, }\StringTok{"X.Pathway."}\NormalTok{ =}\StringTok{ "Protein folding"}\NormalTok{,}
                   \StringTok{"X.Number.of.Genes."}\NormalTok{ =}\StringTok{ }\KeywordTok{length}\NormalTok{(Protein_folding_overlap),}
                   \StringTok{"X.Frequency."}\NormalTok{ =}\StringTok{ }\NormalTok{Protein_folding_frequency, }\StringTok{"X.Ratio."}\NormalTok{ =}\StringTok{ }\NormalTok{Protein_folding_ratio)}
\NormalTok{overlap <-}\StringTok{ }\KeywordTok{add_row}\NormalTok{(overlap, }\StringTok{"X.Pathway."}\NormalTok{ =}\StringTok{ "TCR signaling"}\NormalTok{,}
                   \StringTok{"X.Number.of.Genes."}\NormalTok{ =}\StringTok{ }\KeywordTok{length}\NormalTok{(TCR_proteins_overlap), }
                   \StringTok{"X.Frequency."}\NormalTok{ =}\StringTok{ }\NormalTok{TCR_frequency, }\StringTok{"X.Ratio."}\NormalTok{ =}\StringTok{ }\NormalTok{TCR_ratio)}
\NormalTok{overlap <-}\StringTok{ }\KeywordTok{add_row}\NormalTok{(overlap, }\StringTok{"X.Pathway."}\NormalTok{ =}\StringTok{ "UPR"}\NormalTok{,}
                   \StringTok{"X.Number.of.Genes."}\NormalTok{ =}\StringTok{ }\KeywordTok{length}\NormalTok{(upr_overlay),}
                   \StringTok{"X.Frequency."}\NormalTok{ =}\StringTok{ }\NormalTok{UPR_frequency,}\StringTok{"X.Ratio."}\NormalTok{ =}\StringTok{ }\NormalTok{UPR_ratio)}
\NormalTok{overlap <-}\StringTok{ }\NormalTok{overlap[}\KeywordTok{c}\NormalTok{(}\OperatorTok{-}\DecValTok{1}\NormalTok{),]}


\CommentTok{##code below makes bar graph for number of proteins bound for each pathway##}
\NormalTok{Proteins_bound <-}\StringTok{ }\KeywordTok{as.numeric}\NormalTok{(overlap}\OperatorTok{$}\NormalTok{X.Number.of.Genes.)}
\NormalTok{Pathway <-}\StringTok{ }\NormalTok{overlap}\OperatorTok{$}\NormalTok{X.Pathway.}
\NormalTok{figure1 <-}\StringTok{ }\KeywordTok{ggplot}\NormalTok{(}\DataTypeTok{data =}\NormalTok{ overlap, }\KeywordTok{aes}\NormalTok{(}\DataTypeTok{x =}\NormalTok{ Pathway, }\DataTypeTok{y =}\NormalTok{ Proteins_bound)) }\OperatorTok{+}\StringTok{ }
\StringTok{  }\KeywordTok{geom_col}\NormalTok{(}\DataTypeTok{fill =} \StringTok{"steelblue"}\NormalTok{) }\OperatorTok{+}\StringTok{ }\KeywordTok{theme}\NormalTok{(}\DataTypeTok{plot.title =} \KeywordTok{element_text}\NormalTok{(}\DataTypeTok{family =} \StringTok{"Helvetica"}\NormalTok{, }
                                                                 \DataTypeTok{face =} \StringTok{"bold"}\NormalTok{, }
                                                                 \DataTypeTok{hjust =} \FloatTok{0.5}\NormalTok{, }\DataTypeTok{size =} \DecValTok{16}\NormalTok{), }
                                       \DataTypeTok{axis.text.x =} \KeywordTok{element_text}\NormalTok{(}\DataTypeTok{angle =} \DecValTok{35}\NormalTok{, }\DataTypeTok{vjust =} \FloatTok{0.5}\NormalTok{, }\DataTypeTok{size =} \DecValTok{8}\NormalTok{, }
                                                                  \DataTypeTok{face =} \StringTok{"bold"}\NormalTok{)) }\OperatorTok{+}\StringTok{ }
\StringTok{  }\KeywordTok{ggtitle}\NormalTok{(}\StringTok{"Mutant CALR Bound Proteins per Pathway"}\NormalTok{)}
\NormalTok{figure1}
\end{Highlighting}
\end{Shaded}

\includegraphics{binding_partners_final2_files/figure-latex/unnamed-chunk-1-1.pdf}

\begin{Shaded}
\begin{Highlighting}[]
\CommentTok{##bar graph for average binding enhancement (frequency) for each pathway##}
\NormalTok{Binding_enhancement <-}\StringTok{ }\KeywordTok{as.numeric}\NormalTok{(overlap}\OperatorTok{$}\NormalTok{X.Frequency.)}
\NormalTok{figure2 <-}\StringTok{ }\KeywordTok{ggplot}\NormalTok{(}\DataTypeTok{data =}\NormalTok{ overlap, }\KeywordTok{aes}\NormalTok{(}\DataTypeTok{x =}\NormalTok{ Pathway, }\DataTypeTok{y =}\NormalTok{ Binding_enhancement)) }\OperatorTok{+}\StringTok{ }
\StringTok{  }\KeywordTok{geom_col}\NormalTok{(}\DataTypeTok{fill =} \StringTok{"steelblue"}\NormalTok{) }\OperatorTok{+}\StringTok{ }\KeywordTok{theme}\NormalTok{(}\DataTypeTok{plot.title =} \KeywordTok{element_text}\NormalTok{(}\DataTypeTok{family =} \StringTok{"Helvetica"}\NormalTok{, }\DataTypeTok{face =} \StringTok{"bold"}\NormalTok{, }
                                                                 \DataTypeTok{hjust =} \FloatTok{0.5}\NormalTok{, }\DataTypeTok{size =} \DecValTok{16}\NormalTok{),}
                                       \DataTypeTok{axis.text.x =} \KeywordTok{element_text}\NormalTok{(}\DataTypeTok{angle =} \DecValTok{35}\NormalTok{, }\DataTypeTok{vjust =} \FloatTok{0.5}\NormalTok{, }\DataTypeTok{size =} \DecValTok{8}\NormalTok{, }\DataTypeTok{face =} \StringTok{"bold"}\NormalTok{)) }\OperatorTok{+}\StringTok{ }
\StringTok{  }\KeywordTok{ggtitle}\NormalTok{(}\StringTok{"Average Mutant CALR Binding frequency per Pathway"}\NormalTok{)}
\NormalTok{figure2}
\end{Highlighting}
\end{Shaded}

\begin{verbatim}
## Warning: Removed 1 rows containing missing values (position_stack).
\end{verbatim}

\includegraphics{binding_partners_final2_files/figure-latex/unnamed-chunk-1-2.pdf}

\begin{Shaded}
\begin{Highlighting}[]
\CommentTok{##Bar graph showing ratio of proteins bound/ total proteins in pathway##}
\NormalTok{Ratio_of_proteins_bound <-}\StringTok{ }\KeywordTok{as.numeric}\NormalTok{(overlap}\OperatorTok{$}\NormalTok{X.Ratio.)}
\NormalTok{figure3 <-}\StringTok{ }\KeywordTok{ggplot}\NormalTok{(}\DataTypeTok{data =}\NormalTok{ overlap, }\KeywordTok{aes}\NormalTok{(}\DataTypeTok{x =}\NormalTok{ Pathway, }\DataTypeTok{y =}\NormalTok{ Ratio_of_proteins_bound)) }\OperatorTok{+}\StringTok{ }
\StringTok{  }\KeywordTok{geom_col}\NormalTok{(}\DataTypeTok{fill =} \StringTok{"steelblue"}\NormalTok{) }\OperatorTok{+}\StringTok{ }\KeywordTok{theme}\NormalTok{(}\DataTypeTok{plot.title =} \KeywordTok{element_text}\NormalTok{(}\DataTypeTok{family =} \StringTok{"Helvetica"}\NormalTok{, }\DataTypeTok{face =} \StringTok{"bold"}\NormalTok{, }
                                                                 \DataTypeTok{hjust =} \FloatTok{0.5}\NormalTok{, }\DataTypeTok{size =} \DecValTok{16}\NormalTok{),}
                                       \DataTypeTok{axis.text.x =} \KeywordTok{element_text}\NormalTok{(}\DataTypeTok{angle =} \DecValTok{45}\NormalTok{, }\DataTypeTok{vjust =} \FloatTok{0.5}\NormalTok{, }
                                                                  \DataTypeTok{size =} \DecValTok{8}\NormalTok{, }\DataTypeTok{face =} \StringTok{"bold"}\NormalTok{)) }\OperatorTok{+}\StringTok{ }
\StringTok{  }\KeywordTok{ggtitle}\NormalTok{(}\StringTok{"Percent of Pathway bound by mutant CALR"}\NormalTok{)}
\NormalTok{figure3}
\end{Highlighting}
\end{Shaded}

\includegraphics{binding_partners_final2_files/figure-latex/unnamed-chunk-1-3.pdf}

\begin{Shaded}
\begin{Highlighting}[]
\CommentTok{##dot plot comparing frequency of binding to number of proteins bound}

\KeywordTok{library}\NormalTok{(ggpubr)}
\end{Highlighting}
\end{Shaded}

\begin{verbatim}
## Loading required package: magrittr
\end{verbatim}

\begin{verbatim}
## 
## Attaching package: 'magrittr'
\end{verbatim}

\begin{verbatim}
## The following object is masked from 'package:tidyr':
## 
##     extract
\end{verbatim}

\begin{Shaded}
\begin{Highlighting}[]
\NormalTok{figure4 <-}\StringTok{ }\KeywordTok{ggplot}\NormalTok{(}\DataTypeTok{data =}\NormalTok{ overlap, }\KeywordTok{aes}\NormalTok{(}\DataTypeTok{x =}\NormalTok{ Proteins_bound, }\DataTypeTok{y =}\NormalTok{ Binding_enhancement)) }\OperatorTok{+}
\StringTok{  }\KeywordTok{geom_point}\NormalTok{() }\OperatorTok{+}\StringTok{ }\KeywordTok{theme}\NormalTok{(}\DataTypeTok{plot.title =} \KeywordTok{element_text}\NormalTok{(}\DataTypeTok{family =} \StringTok{"Helvetica"}\NormalTok{, }
                                                 \DataTypeTok{face =} \StringTok{"bold"}\NormalTok{, }\DataTypeTok{hjust =} \FloatTok{0.5}\NormalTok{, }\DataTypeTok{size =} \DecValTok{16}\NormalTok{),}
                       \DataTypeTok{axis.text.x =} \KeywordTok{element_text}\NormalTok{(}\DataTypeTok{angle =} \DecValTok{35}\NormalTok{, }\DataTypeTok{vjust =} \FloatTok{0.5}\NormalTok{, }\DataTypeTok{size =} \DecValTok{8}\NormalTok{, }\DataTypeTok{face =} \StringTok{"bold"}\NormalTok{)) }\OperatorTok{+}
\StringTok{  }\KeywordTok{stat_cor}\NormalTok{(}\DataTypeTok{method =} \StringTok{"pearson"}\NormalTok{, }\DataTypeTok{label.x =} \DecValTok{28}\NormalTok{, }\DataTypeTok{label.y =} \DecValTok{20}\NormalTok{) }\OperatorTok{+}
\StringTok{  }\KeywordTok{ggtitle}\NormalTok{(}\StringTok{"Bound Proteins vs. Binding enhancement"}\NormalTok{)}
\NormalTok{figure4}
\end{Highlighting}
\end{Shaded}

\begin{verbatim}
## Warning: Removed 1 rows containing non-finite values (stat_cor).
\end{verbatim}

\begin{verbatim}
## Warning: Removed 1 rows containing missing values (geom_point).
\end{verbatim}

\includegraphics{binding_partners_final2_files/figure-latex/unnamed-chunk-1-4.pdf}

\begin{Shaded}
\begin{Highlighting}[]
\CommentTok{##this figure is just for fun, but proves there is very little correlation between the number}
\CommentTok{## of proteins bound and the strength with which they bind##}


\CommentTok{##compile all bound proteins from these pathways in order to identify proteins that appear in }
\CommentTok{##multiple pathways. First section pulls entire row of data from the original binding_partners}
\CommentTok{##dataset, while the second part stacks each pull to form a complete data frame}
\NormalTok{binding_partners1 <-}\StringTok{ }\NormalTok{binding_partners[,}\DecValTok{1}\OperatorTok{:}\DecValTok{3}\NormalTok{]}
\NormalTok{glycolysis_}\DecValTok{1}\NormalTok{ <-}\StringTok{ }\NormalTok{binding_partners1[Glycolysis_numbers,]}
\NormalTok{glycolysis_}\DecValTok{1}\OperatorTok{$}\StringTok{"pathway"}\NormalTok{[}\DecValTok{1}\OperatorTok{:}\DecValTok{4}\NormalTok{] =}\StringTok{ 'glycolysis'}
\NormalTok{AAMetabolism_}\DecValTok{1}\NormalTok{ <-}\StringTok{ }\NormalTok{binding_partners1[AAMetabolism_numbers,] }
\NormalTok{AAMetabolism_}\DecValTok{1}\OperatorTok{$}\StringTok{"pathway"}\NormalTok{[}\DecValTok{1}\OperatorTok{:}\DecValTok{27}\NormalTok{] =}\StringTok{ "amino acid metabolism"}
\NormalTok{AATransport_}\DecValTok{1}\NormalTok{ <-}\StringTok{ }\NormalTok{binding_partners1[AATransport_numbers,]}
\NormalTok{AATransport_}\DecValTok{1}\OperatorTok{$}\StringTok{"pathway"}\NormalTok{[}\DecValTok{1}\OperatorTok{:}\DecValTok{7}\NormalTok{] =}\StringTok{ "amino acid transporters"}
\NormalTok{Antigen_processing_}\DecValTok{1}\NormalTok{ <-}\StringTok{ }\NormalTok{binding_partners1[Antigen_processing_numbers,]}
\NormalTok{Antigen_processing_}\DecValTok{1}\OperatorTok{$}\StringTok{"pathway"}\NormalTok{[}\DecValTok{1}\OperatorTok{:}\DecValTok{7}\NormalTok{] =}\StringTok{ "antigen processing"}
\NormalTok{Apoptosis_}\DecValTok{1}\NormalTok{ <-}\StringTok{ }\NormalTok{binding_partners1[Apoptosis_numbers,]}
\NormalTok{Apoptosis_}\DecValTok{1}\OperatorTok{$}\StringTok{"pathway"}\NormalTok{[}\DecValTok{1}\OperatorTok{:}\DecValTok{44}\NormalTok{] =}\StringTok{ "Apoptosis"}
\NormalTok{ATF4_}\DecValTok{1}\NormalTok{ <-}\StringTok{ }\NormalTok{binding_partners1[ATF4_numbers,]}
\NormalTok{ATF4_}\DecValTok{1}\OperatorTok{$}\StringTok{"pathway"}\NormalTok{[}\DecValTok{1}\OperatorTok{:}\DecValTok{6}\NormalTok{] =}\StringTok{ "ATF4 binding"}
\NormalTok{mTOR_}\DecValTok{1}\NormalTok{ <-}\StringTok{ }\NormalTok{binding_partners1[mTOR_numbers,]}
\NormalTok{mTOR_}\DecValTok{1}\OperatorTok{$}\StringTok{"pathway"}\NormalTok{[}\DecValTok{1}\OperatorTok{:}\DecValTok{4}\NormalTok{] =}\StringTok{ "mTOR signaling"}
\NormalTok{NFkB_}\DecValTok{1}\NormalTok{ <-}\StringTok{ }\NormalTok{binding_partners1[NFkB_numbers,]}
\NormalTok{NFkB_}\DecValTok{1}\OperatorTok{$}\StringTok{"pathway"}\NormalTok{[}\DecValTok{1}\OperatorTok{:}\DecValTok{26}\NormalTok{] =}\StringTok{ "NFkB signaling"}
\NormalTok{nucleosome_proteins_}\DecValTok{1}\NormalTok{ <-}\StringTok{ }\NormalTok{binding_partners1[nucleosome_numbers,]}
\NormalTok{nucleosome_proteins_}\DecValTok{1}\OperatorTok{$}\StringTok{"pathway"}\NormalTok{[}\DecValTok{1}\OperatorTok{:}\DecValTok{4}\NormalTok{] =}\StringTok{ "Nucleosome formation"}
\NormalTok{p53_indep_}\DecValTok{1}\NormalTok{ <-}\StringTok{ }\NormalTok{binding_partners1[p53_numbers,]}
\NormalTok{p53_indep_}\DecValTok{1}\OperatorTok{$}\StringTok{"pathway"}\NormalTok{[}\DecValTok{1}\OperatorTok{:}\DecValTok{7}\NormalTok{] =}\StringTok{ "p53 independent DNA damage and repair"}
\NormalTok{PD1_}\DecValTok{1}\NormalTok{ <-}\StringTok{ }\NormalTok{binding_partners1[PD1_numbers,]}
\NormalTok{PD1_}\DecValTok{1}\OperatorTok{$}\StringTok{"pathway"}\NormalTok{[}\DecValTok{1}\OperatorTok{:}\DecValTok{2}\NormalTok{] =}\StringTok{ "PD1 signaling"}
\NormalTok{protein_folding_}\DecValTok{1}\NormalTok{ <-}\StringTok{ }\NormalTok{binding_partners1[Protein_folding_numbers,]}
\NormalTok{protein_folding_}\DecValTok{1}\OperatorTok{$}\StringTok{"pathway"}\NormalTok{[}\DecValTok{1}\OperatorTok{:}\DecValTok{14}\NormalTok{] =}\StringTok{ "protein folding proteins"}
\NormalTok{TCR_}\DecValTok{1}\NormalTok{ <-}\StringTok{ }\NormalTok{binding_partners1[TCR_numbers,]}
\NormalTok{TCR_}\DecValTok{1}\OperatorTok{$}\StringTok{"pathway"}\NormalTok{[}\DecValTok{1}\OperatorTok{:}\DecValTok{12}\NormalTok{] =}\StringTok{ "TCR signaling"}
\NormalTok{UPR_}\DecValTok{1}\NormalTok{ <-}\StringTok{ }\NormalTok{binding_partners1[UPR_numbers,]}
\NormalTok{UPR_}\DecValTok{1}\OperatorTok{$}\StringTok{"pathway"}\NormalTok{[}\DecValTok{1}\OperatorTok{:}\DecValTok{17}\NormalTok{] =}\StringTok{ "UPR numbers"}

\CommentTok{##compile each protein using STACK function}
\NormalTok{a <-}\StringTok{ }\KeywordTok{Stack}\NormalTok{(AAMetabolism_}\DecValTok{1}\NormalTok{, AATransport_}\DecValTok{1}\NormalTok{)}
\NormalTok{b <-}\StringTok{ }\KeywordTok{Stack}\NormalTok{(a, Antigen_processing_}\DecValTok{1}\NormalTok{)}
\NormalTok{c <-}\StringTok{ }\KeywordTok{Stack}\NormalTok{(b, glycolysis_}\DecValTok{1}\NormalTok{)}
\NormalTok{d <-}\StringTok{ }\KeywordTok{Stack}\NormalTok{(c, Apoptosis_}\DecValTok{1}\NormalTok{)}
\NormalTok{e <-}\StringTok{ }\KeywordTok{Stack}\NormalTok{(d, ATF4_}\DecValTok{1}\NormalTok{)}
\NormalTok{f <-}\StringTok{ }\KeywordTok{Stack}\NormalTok{(e, mTOR_}\DecValTok{1}\NormalTok{)}
\NormalTok{g <-}\StringTok{ }\KeywordTok{Stack}\NormalTok{(f, NFkB_}\DecValTok{1}\NormalTok{)}
\NormalTok{h <-}\StringTok{ }\KeywordTok{Stack}\NormalTok{(g, nucleosome_proteins_}\DecValTok{1}\NormalTok{)}
\NormalTok{i <-}\StringTok{ }\KeywordTok{Stack}\NormalTok{(h, p53_indep_}\DecValTok{1}\NormalTok{)}
\NormalTok{j <-}\StringTok{ }\KeywordTok{Stack}\NormalTok{(i, PD1_}\DecValTok{1}\NormalTok{)}
\NormalTok{k <-}\StringTok{ }\KeywordTok{Stack}\NormalTok{(j, protein_folding_}\DecValTok{1}\NormalTok{)}
\NormalTok{l <-}\StringTok{ }\KeywordTok{Stack}\NormalTok{(k, TCR_}\DecValTok{1}\NormalTok{)}
\NormalTok{Compiled_binding_}\DecValTok{1}\NormalTok{ <-}\StringTok{ }\KeywordTok{Stack}\NormalTok{(l, UPR_}\DecValTok{1}\NormalTok{)}

\CommentTok{##put them in decreasing order based on frequency}
\NormalTok{compiled_binding_hfrq <-}\StringTok{ }\NormalTok{Compiled_binding_}\DecValTok{1}\NormalTok{[}\KeywordTok{order}\NormalTok{(Compiled_binding_}\DecValTok{1}\OperatorTok{$}\NormalTok{frequency, }\DataTypeTok{decreasing =} \OtherTok{TRUE}\NormalTok{),]}

\CommentTok{##isolate unique genes}
\NormalTok{compiled_binding_hfrq_numbers <-}\StringTok{ }\KeywordTok{match}\NormalTok{(}\KeywordTok{unique}\NormalTok{(compiled_binding_hfrq}\OperatorTok{$}\NormalTok{gene), }
\NormalTok{                                       compiled_binding_hfrq}\OperatorTok{$}\NormalTok{gene, }\DataTypeTok{nomatch =} \OtherTok{FALSE}\NormalTok{)}
\NormalTok{compiled_binding_hfrq <-}\StringTok{ }\NormalTok{compiled_binding_hfrq[compiled_binding_hfrq_numbers,]}


\NormalTok{duplicated_genes <-}\StringTok{ }\NormalTok{Compiled_binding_}\DecValTok{1}\OperatorTok{$}\NormalTok{gene }\OperatorTok\StringTok{ }\KeywordTok{duplicated}\NormalTok{()}
\CommentTok{##note: for each pair of duplicates, one is printed true and the other false}
\NormalTok{duplicated_genes <-}\StringTok{ }\KeywordTok{as.data.frame}\NormalTok{(duplicated_genes)}
\NormalTok{Compiled_binding_}\DecValTok{1}\OperatorTok{$}\NormalTok{duplicates[}\DecValTok{1}\OperatorTok{:}\DecValTok{251}\NormalTok{] =}\StringTok{ }\NormalTok{duplicated_genes}\OperatorTok{$}\NormalTok{duplicated_genes}
\NormalTok{Compiled_binding_}\DecValTok{1}\NormalTok{ <-}\StringTok{ }\KeywordTok{as.data.frame}\NormalTok{(}\KeywordTok{lapply}\NormalTok{(Compiled_binding_}\DecValTok{1}\NormalTok{, unlist))}
\NormalTok{Compiled_binding_}\DecValTok{1}\NormalTok{ <-}\StringTok{ }\NormalTok{Compiled_binding_}\DecValTok{1}\NormalTok{[}\KeywordTok{order}\NormalTok{(Compiled_binding_}\DecValTok{1}\OperatorTok{$}\NormalTok{duplicates),]}
\NormalTok{Compiled_duplicates <-}\StringTok{ }\NormalTok{Compiled_binding_}\DecValTok{1}\NormalTok{[}\DecValTok{152}\OperatorTok{:}\DecValTok{251}\NormalTok{,] }
\CommentTok{##note: genes with duplicates remaining in this list have multiple duplicates}

\CommentTok{##print tables displaying diplicated genes, highest frequency at the top, lowest at the bottom}
\NormalTok{Compiled_duplicates_unique <-}\StringTok{ }\NormalTok{Compiled_duplicates[}\KeywordTok{order}\NormalTok{(Compiled_duplicates}\OperatorTok{$}\NormalTok{frequency, }\DataTypeTok{decreasing =} \OtherTok{TRUE}\NormalTok{),]}
\KeywordTok{view}\NormalTok{(Compiled_duplicates_unique)}
\NormalTok{unique_numbers <-}\StringTok{ }\KeywordTok{match}\NormalTok{(}\KeywordTok{unique}\NormalTok{(Compiled_duplicates_unique}\OperatorTok{$}\NormalTok{gene), }
\NormalTok{                        Compiled_duplicates_unique}\OperatorTok{$}\NormalTok{gene, }\DataTypeTok{nomatch =} \OtherTok{FALSE}\NormalTok{)}

\NormalTok{Compiled_duplicates_unique1 <-}\StringTok{ }\NormalTok{Compiled_duplicates_unique[unique_numbers,]}
\NormalTok{Compiled_duplicates_unique <-}\StringTok{ }\NormalTok{Compiled_duplicates_unique1[}\KeywordTok{order}\NormalTok{(Compiled_duplicates_unique1}\OperatorTok{$}\NormalTok{frequency,}
                                                                \DataTypeTok{decreasing =} \OtherTok{TRUE}\NormalTok{),]}
\CommentTok{##make list not including proteosome proteins bc proteosome just means its getting degraded##}

\NormalTok{Compiled_duplicates_unique_noproteosome <-}\StringTok{ }\KeywordTok{filter}\NormalTok{(Compiled_duplicates_unique, family }\OperatorTok{!=}\StringTok{ "Proteasome"}\NormalTok{)}

\CommentTok{##Have figure1,2,3,4 compiled_binding_hfrq (highest frequency genes in binding set), compiled_duplicates_unique }
\CommentTok{##(list of highest scoring frequency duplicates),}
\CommentTok{##list of duplicates without proteasome##}

\KeywordTok{library}\NormalTok{(grid)}
\KeywordTok{library}\NormalTok{(gridExtra)}
\end{Highlighting}
\end{Shaded}

\begin{verbatim}
## 
## Attaching package: 'gridExtra'
\end{verbatim}

\begin{verbatim}
## The following object is masked from 'package:dplyr':
## 
##     combine
\end{verbatim}

\begin{Shaded}
\begin{Highlighting}[]
\KeywordTok{library}\NormalTok{(data.table)}
\end{Highlighting}
\end{Shaded}

\begin{verbatim}
## 
## Attaching package: 'data.table'
\end{verbatim}

\begin{verbatim}
## The following objects are masked from 'package:dplyr':
## 
##     between, first, last
\end{verbatim}

\begin{Shaded}
\begin{Highlighting}[]
\CommentTok{##Compile tables, remove row names##}
\CommentTok{##Table1 = highest frequency unique bound proteins}
\KeywordTok{rownames}\NormalTok{(compiled_binding_hfrq) <-}\StringTok{ }\OtherTok{NULL}
\KeywordTok{grid.newpage}\NormalTok{()}
\NormalTok{Table1 <-}\StringTok{ }\KeywordTok{grid.table}\NormalTok{(compiled_binding_hfrq[}\DecValTok{1}\OperatorTok{:}\DecValTok{20}\NormalTok{,}\DecValTok{1}\OperatorTok{:}\DecValTok{4}\NormalTok{])}
\end{Highlighting}
\end{Shaded}

\includegraphics{binding_partners_final2_files/figure-latex/unnamed-chunk-1-5.pdf}

\begin{Shaded}
\begin{Highlighting}[]
\CommentTok{##Table2 = highest frequency proteins bound that have at least one duplicate}
\CommentTok{## + proteasome proteins}
\KeywordTok{rownames}\NormalTok{(Compiled_duplicates_unique) <-}\StringTok{ }\OtherTok{NULL}
\KeywordTok{grid.newpage}\NormalTok{()}
\NormalTok{Table2 <-}\StringTok{ }\KeywordTok{grid.table}\NormalTok{(Compiled_duplicates_unique[}\DecValTok{1}\OperatorTok{:}\DecValTok{20}\NormalTok{,}\DecValTok{1}\OperatorTok{:}\DecValTok{4}\NormalTok{],)}
\end{Highlighting}
\end{Shaded}

\includegraphics{binding_partners_final2_files/figure-latex/unnamed-chunk-1-6.pdf}

\begin{Shaded}
\begin{Highlighting}[]
\CommentTok{##Table3 = highest frequency proteins bound that have at least one duplicate }
\CommentTok{## - proteasome proteins, because most bound to proteosome related proteins are likely}
\CommentTok{## just being degraded}
\KeywordTok{rownames}\NormalTok{(Compiled_duplicates_unique_noproteosome) <-}\StringTok{ }\OtherTok{NULL}
\KeywordTok{grid.newpage}\NormalTok{()}
\NormalTok{Table3 <-}\StringTok{ }\KeywordTok{grid.table}\NormalTok{(Compiled_duplicates_unique_noproteosome[}\DecValTok{1}\OperatorTok{:}\DecValTok{20}\NormalTok{,}\DecValTok{1}\OperatorTok{:}\DecValTok{4}\NormalTok{])}
\end{Highlighting}
\end{Shaded}

\includegraphics{binding_partners_final2_files/figure-latex/unnamed-chunk-1-7.pdf}

\begin{Shaded}
\begin{Highlighting}[]
\CommentTok{##TO ACCESS FINAL FIGURES}
\CommentTok{##run "figure1", "figure2", "figure3", "figure4"}
\CommentTok{##run these blocks of code individually for each table:}
\CommentTok{##Table1}
\KeywordTok{grid.newpage}\NormalTok{()}
\NormalTok{Table1 <-}\StringTok{ }\KeywordTok{grid.table}\NormalTok{(compiled_binding_hfrq[}\DecValTok{1}\OperatorTok{:}\DecValTok{20}\NormalTok{,}\DecValTok{1}\OperatorTok{:}\DecValTok{4}\NormalTok{])}
\end{Highlighting}
\end{Shaded}

\includegraphics{binding_partners_final2_files/figure-latex/unnamed-chunk-1-8.pdf}

\begin{Shaded}
\begin{Highlighting}[]
\CommentTok{##Table2}
\KeywordTok{grid.newpage}\NormalTok{()}
\NormalTok{Table2 <-}\StringTok{ }\KeywordTok{grid.table}\NormalTok{(Compiled_duplicates_unique[}\DecValTok{1}\OperatorTok{:}\DecValTok{20}\NormalTok{,}\DecValTok{1}\OperatorTok{:}\DecValTok{4}\NormalTok{],)}
\end{Highlighting}
\end{Shaded}

\includegraphics{binding_partners_final2_files/figure-latex/unnamed-chunk-1-9.pdf}

\begin{Shaded}
\begin{Highlighting}[]
\CommentTok{##Table3}
\KeywordTok{grid.newpage}\NormalTok{()}
\NormalTok{Table3 <-}\StringTok{ }\KeywordTok{grid.table}\NormalTok{(Compiled_duplicates_unique_noproteosome[}\DecValTok{1}\OperatorTok{:}\DecValTok{20}\NormalTok{,}\DecValTok{1}\OperatorTok{:}\DecValTok{4}\NormalTok{])}
\end{Highlighting}
\end{Shaded}

\includegraphics{binding_partners_final2_files/figure-latex/unnamed-chunk-1-10.pdf}


\end{document}
